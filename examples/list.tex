
% Created by Roger Hu's .pytex --> .tex latex transcriber
% Compiled on 2020-09-04 13:14:10 PDT


\documentclass[12pt]{article}
\usepackage{amsmath, amssymb, amsthm, geometry, enumitem, fancyhdr}

\geometry{letterpaper, portrait, margin=1in}
\setlength{\parindent}{0em}
\linespread{1.3}
\pagestyle{fancy}
\fancyhf{}

\renewcommand{\headrulewidth}{0pt}
\renewcommand{\footrulewidth}{1pt}
\rhead{Lists}
\rfoot{\fontsize{8}{8} \selectfont \thepage}

\newtheorem{Theorem}{Theorem}
\newtheorem{Corollary}{Corollary}[Theorem]
\newtheorem{Lemma}[Theorem]{Lemma}

\begin{document}

\begin{flushleft}

Use \verb|..begin list bullet [label]| or \verb|..begin list [label]| to create a list.

[label] can be any value for a bullet point list, but for now, for an ordered list, the characters `a', `i', and `1', represent ordering alphabetically, by roman numerals, and arabic numerals, respectively.

\verb|..begin list bullet|\\
\verb|.. Item 1|\\
\verb|.. Item 2|\\
\verb|.. Item 3|\\

\begin{itemize}
    \item Item 1
    \item Item 2
    \item Item 3
\end{itemize}

\verb|..begin list bullet *|\\
\verb|.. Item 1|\\
\verb|.. Item 2|\\
\verb|.. Item 3|\\

\begin{itemize}[label=*]
    \item Item 1
    \item Item 2
    \item Item 3
\end{itemize}

\verb|..begin list a.|\\
\verb|.. Item 1|\\
\verb|.. Item 2|\\
\verb|.. Item 3|\\

\begin{enumerate}[label=\alph*.]
    \item Item 1
    \item Item 2
    \item Item 3
\end{enumerate}

\verb|..begin list (i)|\\
\verb|.. Item 1|\\
\verb|.. Item 2|\\
\verb|.. Item 3|\\

\begin{enumerate}[label=(\roman*)]
    \item Item 1
    \item Item 2
    \item Item 3
\end{enumerate}

\verb|..begin list 1)|\\
\verb|.. Item 1|\\
\verb|.. Item 2|\\
\verb|.. Item 3|\\

\begin{enumerate}[label=\arabic*)]
    \item Item 1
    \item Item 2
    \item Item 3
\end{enumerate}
\end{flushleft}

\end{document}
