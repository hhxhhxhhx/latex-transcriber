
% Created by Roger Hu's .pytex --> .tex latex transcriber
% Compiled on 2020-09-29 22:40:17 PDT


\documentclass[12pt]{article}
\usepackage{amsmath, amssymb, amsthm, geometry, enumitem, fancyhdr}

\geometry{letterpaper, portrait, margin=1in}
\setlength{\parindent}{0em}
\linespread{1.3}
\pagestyle{fancy}
\fancyhf{}

\newcommand{\nl}{\\}

\renewcommand{\headrulewidth}{0pt}
\renewcommand{\footrulewidth}{1pt}
\rhead{TFDS}
\rfoot{\fontsize{8}{8} \selectfont \thepage}

\newtheorem{Theorem}{Theorem}
\newtheorem{Corollary}{Corollary}[Theorem]
\newtheorem{Lemma}[equation]{Lemma}
\numberwithin{equation}{section}

\begin{document}

\begin{flushleft}

\verb=..begin tfds l|l|c|l|l=\nl
\verb=[1] \ p \supset q \ P=\nl
\verb=[2] \ -q \ P=\nl
\verb=[1,2] \ -p \ (1)(2) \ MT=\nl
\verb=[1] \ -q \supset -p \ [2](3) \ D=\nl
\verb=..end tfds=\nl
\bigskip

$\begin{array}{l|l|c|l|l}
{[1]} & (1) & p \supset q & & \text{P} \\
{[2]} & (2) & -q & & \text{P} \\
{[1,2]} & (3) & -p & (1)(2) & \text{MT} \\
{[1]} & (4) & -q \supset -p & [2](3) & \text{D} 
\end{array}$

\bigskip

In the example above, the first line would normally have to be written as \verb=[1] \ p \supset q \ \ P=, because 4 column breaks is expected. However, if that line is a premise (the line ends with a ` P'), the \verb|\| or \verb|&| can be ommitted.

\bigskip

\verb|..begin tfds cccc|\nl
\verb|[1] & p \supset q & P|\nl
\verb|[2] & -q & P|\nl
\verb|[1,2] & -p & (1)(2) MT|\nl
\verb|[1] & -q \supset -p & [2](3) D|\nl
\verb|[] & (p \supset q) \supset (-q \supset -p) & [1](4) D|\nl
\verb|..end tfds|\nl
\bigskip

$\begin{array}{cccc}
{[1]} & (1) & p \supset q & \text{P} \\
{[2]} & (2) & -q & \text{P} \\
{[1,2]} & (3) & -p & (1)(2) \text{MT} \\
{[1]} & (4) & -q \supset -p & [2](3) \text{D} \\
{[~]} & (5) & (p \supset q) \supset (-q \supset -p) & [1](4) \text{D} 
\end{array}$

\bigskip

Here's some other text.
\end{flushleft}

\end{document}
