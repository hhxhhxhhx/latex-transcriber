
% Created by Roger Hu's .pytex --> .tex latex transcriber
% Compiled on 2020-09-04 13:26:13 PDT


\documentclass[12pt]{article}
\usepackage{amsmath, amssymb, amsthm, geometry, enumitem, fancyhdr}

\geometry{letterpaper, portrait, margin=1in}
\setlength{\parindent}{0em}
\linespread{1.3}
\pagestyle{fancy}
\fancyhf{}

\renewcommand{\headrulewidth}{0pt}
\renewcommand{\footrulewidth}{1pt}
\rhead{Tables}
\rfoot{\fontsize{8}{8} \selectfont \thepage}

\newtheorem{Theorem}{Theorem}
\newtheorem{Corollary}{Corollary}[Theorem]
\newtheorem{Lemma}[Theorem]{Lemma}

\begin{document}

\begin{flushleft}

Regular tables (tabular), in non-math mode.

\verb=..begin table c|c|c= \\
\verb|\ Names \| \\
\verb|\hline| \\
\verb|John \ Mary \ Jane| \\
\verb|Adam \ Kevin \ Allie| \\
\verb|..end table|

\bigskip
\bigskip

\begin{tabular}{c|c|c}
& Names & \\
\hline
John & Mary & Jane \\
Adam & Kevin & Allie 
\end{tabular}

\bigskip
\bigskip
\bigskip
\bigskip

Arrays (arrays), in math mode.

\verb=..begin table math r|c|l|= \\
\verb|2 \ 3 \ 465| \\
\verb|\hline| \\
\verb|p |\verb|..==> q \ q \implies r \ 6| \\
\verb|..end table|

\bigskip
\bigskip

$\begin{array}{r|c|l|}
2 & 3 & 4 \\
\hline
p \implies q & q \implies r & 6 
\end{array}$

\end{flushleft}

\end{document}
