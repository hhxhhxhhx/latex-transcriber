
% Created by Roger Hu's .pytex --> .tex latex transcriber
% Compiled on 2020-09-04 13:56:25 PDT


\documentclass[12pt]{article}
\usepackage{amsmath, amssymb, amsthm, geometry, enumitem, fancyhdr}

\geometry{letterpaper, portrait, margin=1in}
\setlength{\parindent}{0em}
\linespread{1.6}
\pagestyle{fancy}
\fancyhf{}

\renewcommand{\headrulewidth}{0pt}
\renewcommand{\footrulewidth}{1pt}
\rhead{Roger Hu}
\rfoot{\fontsize{8}{8} \selectfont \thepage}

\newtheorem{Theorem}{Theorem}
\newtheorem{Corollary}{Corollary}[Theorem]
\newtheorem{Lemma}[Theorem]{Lemma}

\begin{document}

\begin{flushleft}

By default, theorems, lemmas, and corollaries are defined as follows: \\
\verb|\newtheorem{Theorem}{Theorem}|\\
\verb|\newtheorem{Corollary}{Corollary}[Theorem]|\\
\verb|\newtheorem{Lemma}[Theorem]{Lemma}|\\

\bigskip

To change these defaults, or add others, check the next page for uses for \verb|..initheorem*| or \verb|..initheorem| like so.

\bigskip

To begin a theorem environment, use \verb|..begin thm Theorem_Type [Theorem name]|. Theorem types that are available by default are ``Theorem", ``Lemma", and ``Corollary".

\bigskip
\bigskip

\verb|..begin thm Theorem Stoke's Theorem|
\begin{Theorem}[Stoke's Theorem]
For a closed surface oriented counter-clockwise,
\begin{gather*}
\int\limits_C \vec{F} \cdot d\vec{r} = \iint\limits_S (\nabla \times \vec{F}) \cdot d\textbf{S} 
\end{gather*}
\end{Theorem}
\verb|..end thm Theorem|

\bigskip
\bigskip

\verb|..begin thm Lemma|
\begin{Lemma}
This is a lemma.
\end{Lemma}
\verb|..end thm Lemma|

\newpage

\verb|..initheorem* XYZ| will create theorem type \textbf{XYZ} with no numbering. Equivalent to \verb|\newtheorem*{XYZ}{XYZ}|\\
\verb|..initheorem XYZ| will create theorem type \textbf{XYZ} with a counter that increments each time \textbf{XYZ} is created. Equivalent to \verb|\newtheorem{XYZ}{XYZ}|\\
\verb|..initheorem XYZ ABC sub| will create theorem type \textbf{XYZ} that is \textbf{subordinate} to counter type \textbf{ABC}. Equivalent to \verb|\newtheorem{XYZ}{XYZ}[ABC]|\\
\verb|..initheorem XYZ ABC shared| will create theorem type \textbf{XYZ} that is \textbf{shared} with counter type \textbf{ABC}. Equivalent to \verb|\newtheorem{XYZ}[ABC]{XYZ}|\\

\end{flushleft}

\end{document}
