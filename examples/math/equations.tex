
% Created by Roger Hu's .pytex --> .tex latex transcriber
% Compiled on 2020-09-08 21:01:51 PDT


\documentclass[12pt]{article}
\usepackage{amsmath, amssymb, amsthm, geometry, enumitem, fancyhdr}

\geometry{letterpaper, portrait, margin=1in}
\setlength{\parindent}{0em}
\linespread{1.3}
\pagestyle{fancy}
\fancyhf{}

\renewcommand{\headrulewidth}{0pt}
\renewcommand{\footrulewidth}{1pt}
\rhead{Equations}
\rfoot{\fontsize{8}{8} \selectfont \thepage}

\newtheorem{Theorem}{Theorem}
\newtheorem{Corollary}{Corollary}[Theorem]
\newtheorem{Lemma}[Theorem]{Lemma}

\begin{document}

\begin{flushleft}
Here's the \textbf{gather} mode with no numbering: \\
\verb|..begin eq|\\
\verb|a = b + c|\\
\verb|d = e + f|\\
\verb|a + d = b + c|\\
\verb|..end eq|

\begin{gather*}
a = b + c \\
d = e + f \\
a + d = b + c 
\end{gather*}

\medskip

Here's the \textbf{align} mode with no numbering: \\
\verb|..begin eq align|\\
\verb|h &= 2x + 3y + 4z     &   2u + &2v = 10|\\
\verb|x &= y                &   &2v + 4w = 20|\\
\verb|a &= b + c            &   2b + 2c + &7v = 30|\\
\verb|f(x, y) &= g(a, b, c) &   &2v = 4z|\\
\verb|..end eq|

\begin{align*}
h &= 2x + 3y + 4z           &   2u + &2v = 10 \\
x &= y                      &   &2v + 4w = 20 \\
a &= b + c                  &   2b + 2c + &7v = 30 \\
f(x, y) &= g(a, b, c)       &   &2v = 4z 
\end{align*}

\medskip

Here's the \textbf{gather} mode with numbering: \\
\verb|..begin eq number|\\
\verb|a = b|\\
\verb|2x + 4c = 10|\\
\verb|10 + 4c = 20|\\
\verb|..end eq|
\begin{gather}
a = b \\
2x + 4c = 10 \\
10 + 4c = 20 
\end{gather}

\medskip

Here's how to do \textbf{align} with numbering: \\
\verb|..begin eq align number|\\
\verb|a &= b + c|\\
\verb|a &= 2b + d|\\
\verb|a &= 3c + e|\\
\verb|..end eq|
\begin{align}
a &= b + c \\
a &= 2b + d \\
a &= 3c + e 
\end{align}

\end{flushleft}

\end{document}
