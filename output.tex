
\documentclass[12pt]{article}
\usepackage{amsmath, amssymb, amsthm, geometry}

\geometry{letterpaper, portrait, margin=1in}
\setlength{\parindent}{0em}
\linespread{1.3}

\newcommand{\norm}[1]{\|#1\|}
\renewcommand\qedsymbol{Q.E.D}

\newtheorem*{Theorem}{Theorem}
\newtheorem{Lemma}{Lemma}
\newtheorem{Corollary}{Corollary}[Lemma]
\newtheorem{Note}[Lemma]{Note}

\begin{document}

To create a type of theorem, be it \emph{Theorem}, \emph{Lemma}, \emph{Corollary}, whatever, do this.

\begin{Theorem}
This is the contents of Theorem. Notice that it does not have a counter.
\end{Theorem}

That means it doesn't matter how many \emph{Theorem}s you have, they'll always show up as \textbf{Theorem}.

\begin{Theorem}
Here is another theorem.
\end{Theorem}

\begin{Theorem}
And another.
\end{Theorem}

On the other hand, \emph{Lemma}s have a counter, since it was defined without the asterisk.

\begin{Lemma}
Yes, I know this isn't a lemma, but see how there's a counter?
\end{Lemma}

And since \emph{Note} shares the same counter as \emph{Lemma}, the counter will change whenever we define
a new \emph{Note} or \emph{Lemma}.

\begin{Note}
Hello World! New counter!
\end{Note}

\begin{Lemma}
Counter changes again!
\end{Lemma}

However, \emph{Corollary}s are defined to be subordinate to \emph{Lemma} counters, take a look.

\begin{Corollary}
Hello, I'm a corollary.
\end{Corollary}

\begin{Corollary}
What a coincidence, me too!
\end{Corollary}

Once the "superior" counter changes, in this case \emph{Lemma}, \emph{Corollary}s' counter will also change.

\begin{Note}
Our counter changed!
\end{Note}

\begin{Corollary}
It should be self-explanatory why my counter is the way it is now.
\end{Corollary}

We can also give names to theorems, like this.

\begin{Theorem}[Stoke's Theorem]
For a closed surface oriented counter-clockwise,
\[ \int\limits_C \vec{F} \cdot d\vec{r} = \iint\limits_S (\nabla \times \vec{F}) \cdot d\textbf{S} \]
\end{Theorem}

\begin{Note}[Some arbitrary title]
It works for \emph{Note}, \emph{Lemma}, etc.
\end{Note}

It's extremely simple to create a proof.

\medskip

\textbf{Claim:} an irrational number raised to an irrational power can be rational.
..beginproof

Notice that $\sqrt{2} \in \mathbb{R}$, $\notin \mathbb{Q}$. Therefore, we know that $\sqrt{2}^{\sqrt{2}} \in \mathbb{R}$ but can be $\in \mathbb{Q}$ or $\notin \mathbb{Q}$.

Suppose $\sqrt{2}^{\sqrt{2}} \in \mathbb{Q}$, we're done.

Suppose $\sqrt{2}^{\sqrt{2}} \notin \mathbb{Q}$, we can then use this value and raise it to another power of $\sqrt{2}$.

\[ (\sqrt{2}^{\sqrt{2}})^{\sqrt{2}} = \sqrt{2}^{\sqrt{2} * \sqrt{2}} = \sqrt{2}^2 = 2 \in \mathbb{Q} \]
..endproof


Do the following to create a bullet point list (unnumbered).
If the type of list is not specified to be "bullet", it'll be defaulted to be
ordered.
\begin{itemize}
    \item item 1.
    \item item 2.
    this will go to the end of item 2, and not be in a new line, because it's missing the two dots in the beginning.
    \item item 3.
    \\ item 3.5, notice that this doesn't have a bullet in front of it.
    \item item 4.
    \begin{enumerate}
        \item nested lists are also allowed!
        \item this one is ordered.
        \begin{enumerate}
            \item and another nested list!
        \end{enumerate}
        \item just remember to end as many lists as you started.
    \end{enumerate}
    
    
    or it'll be embarassing, just like this here...
    
\end{itemize}

Oops, wrong indentation!

Or better yet, it'll transcribe correctly,
but will not \textbf{compile} correctly if an \emph{..endlist} is missing!

It might be helpful to indent the lists in the code, so you won't forget what level of indentation you're on. LateX is great in the sense that indentation doesn't matter \textbf{that} much.

\begin{Note}
The transcribed .tex files will automatically be indented wherever there are items in a list.
\end{Note}

\end{document}

% Created by python -> latex autotranscriber
